% \documentclass[UTF8,a4paper]{ctexart}

\documentclass{report}
\usepackage[UTF8]{ctex}
\usepackage{titlesec} % 标题格式
\usepackage[top=4cm, bottom=2.5cm, left=3.5cm, right=2.5cm]{geometry} % 间隙


\renewcommand{\baselinestretch}{1.5} % 行距

\titleclass{\chapter}{straight} % 取消自动分页
\titleformat{\chapter}{\Large\bfseries}{\thechapter.}{1em}{}
\titleformat{\section}{\large\bfseries}{\thesection}{1em}{}
\titleformat{\subsection}{\large\bfseries}{\thesubsection}{1em}{}
\titlespacing*{\chapter}{0pt}{80pt}{10pt} % 间距
\titlespacing*{\section}{0pt}{40pt}{15pt} % 间距


% P0
\title{\Huge \bfseries \vspace*{1cm} 月度学习报告}
\author{
    \Large
    \begin{tabular}{r @{\quad} l} 
        \textbf{姓\qquad 名:} & 刘世豪 \\
        \textbf{学\qquad 号:} & 22560439 \\
        \textbf{专\qquad 业:} & 大数据技术与工程 \\
    \end{tabular}
}
\date{2025年11月} 



\begin{document}
\maketitle

% 1
\chapter{概述}
本月,我完成了对数值代数的学习。数值代数主要为了解决以下几大类问题:求解线性方程组Ax=b;
在方程组无解时求近似解;求解矩阵的特征值或奇异值。


% 2
\chapter{学习情况}
% \section{具体学习情况}
在求解线性方程组问题中,可以用多种矩阵分解方法简化运算,如
LU分解($A=LU$或$PA=LU$)、
Cholesky分解(在LU分解的基础上将对称矩阵分解$A=KK^T$)、
QR分解(将m$\times$n大小的矩阵分解为正交矩阵和上三角矩阵的积:$A=Q_{m \times m}R_{m \times n}$)、
奇异值分解($A=U_{m \times n}\Sigma_{n \times n}V_{n \times n}$)
等方法。
如果方程组维度较高,则可以用迭代法求近似解从而简化计算,如Jacobi迭代法、Gauss-Seidel迭代法和松弛迭代法。\\

当方程组无解时,可以通过最小二乘法求最优近似解,并用QR分解、SVD分解等方法帮助求解最小二乘法。\\

矩阵的特征向量可以理解为在线性变换中方向保持不变的向量,特征值可以理解为这个方向上的拉伸倍数。
求矩阵的特征值或奇异值时,可以用QR算法、SVD分解、施密特正交化等方法。


% 3
\chapter{未来学习计划}
我计划于下个月完成数值分析的学习。之后,我将从以下几个方向中选择部分作为近期的学习目标:\\
1、进一步学习机器学习的内容,并在kaggle平台上尝试完成一些基础项目。\\
2、继续学习数学课程,并对已学内容进行复习和深入。\\
3、学习C++编程和linxv开发。



\end{document}